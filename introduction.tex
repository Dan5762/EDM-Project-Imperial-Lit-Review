\chapter{Introduction}
\label{introduction}

For the last ${\sim}50$ years Standard Model predictions have been in agreement with experimental observations. However, despite this agreement it is a widely held view that the Standard Model is not a full description of our universe. Instead there are many beyond the Standard Model (BSM) theories which seek to replace the  arbitrariness and seeming incompleteness of the Standard Model, in many cases predicting families of particles, so far unobserved due to their extreme energies. One approach to testing these predictions is through particle colliders such as the Large Hadron Collider (LHC), which attempt to probe these extreme regimes, with the latest experiments reaching energies of up to 14 TeV. However, understanding extreme phenomena does not necessarily require experiments to recreate these extreme energies, instead since the 1950s electric dipole moment (EDM) experiments have been pursued, typically operating at less than a few hundred kelvin.

Experimental searches for permanent particle EDMs stem from their inherent violation of parity and time symmetries, violations which may play a role in many currently unexplained phenomena, such as the matter-antimatter imbalance of our universe. An observation of a non-zero particle EDM would not only provide experimental observation of these asymmetries, but it would also provide a key constraint to models of particle physics, potentially presenting the first experimental observation not accounted for in the standard model.

With this context in mind, the recent advances in experimental precision and proposed future experimental advances are of particular significance in the study of fundamental physics. In this review an overview of an electron EDM (eEDM) and its implications for particle physics is presented in section \ref{background}, the key theoretical tools used in the discussed eEDM experiments are described in section \ref{theory} and finally there is a discussion of the leading experiments from the last 20 years, as well as an introduction to the proposed future experiments are given in section \ref{modern_experiments}.